
\subsection*{\fullwidth{\Large \centering \textbf{General Knowledge Multiple Choice (Set 1)}}}

\begin{questions}

%% in order to provide indication for marks of each question, do it as follows:
% \question[1] What is the relationship between latitude and temperature ?

% conventionally, \begin{oneparchoices} was used to fit multiple choices in the same line

\question What is the relationship between latitude and temperature ?
  \begin{enumerate}
  \item With increase in latitude, temperature increases
  \item With increase in latitude, temperature decreases
  \item With decrease in latitude, temperature increases
  \item With decrease in latitude, temperature decreases
  \end{enumerate}

  \begin{items}
  \item 1 and 3 are correct
  \item 1 and 2 are correct
  \item* 2 and 3 are correct
  \item 3 and 4 are correct
  \end{items}

\question Large and bright meteors are called:
  % this places all choices in separate lines
  \begin{items}
  \item Falling star
  \item Shooting star
  \item* Fire ball
  \item Shooting ball
  \end{items}

\question The country with most languages spoken is:
  % this places all choices in the same line

  \begin{items}
  \item China
  \item* India
  \item America
  \item Russia
  \end{items}

\question Which is the biggest of the deserts ?

  \begin{items}
  \item Somali
  \item* Karakoram
  \item Thar
  \item Attacama
  \end{items}

\question Popular piligrimage Muktinath occurs in the elevation of:
  \begin{items}
  \item 4500 m
  \item* 3750 m
  \item 3500 m
  \item 4100 m
  \end{items}

\question Which amongst these nations have been elected in UN security council for most number of times?
  \begin{items}
  \item Japan
  \item Spain
  \item Nepal
  \item* Uruguay
  \end{items}

\question Chess is the national game of:

  \begin{items}
  \item India
  \item* Russia
  \item Saudi Arabia
  \item none of above
  \end{items}

\question Nepal is not the earliest among south asian nations in:
  \begin{items}
  \item Maintaining diplomatic policy with Israel
  \item* Incorporating "Rights to information" in constitution
  \item Formulating a separate law regarding "Rights to information"
  \item Summiting Mt. Everest.
  \end{items}

\question Which among the below is known as the city of handcrafts:
  \begin{items}
  \item Bhaktapur
  \item* Patan
  \item Bungmati
  \item Bandipur
  \end{items}

\question Nepal is the \fillin[][2cm] nation to have issued constitution through a constitutional assembly.
  \begin{items}
  \item 28
  \item 35
  \item* 44
  \item 53
  \end{items}

\question Which country among the following is the first to grant voting rights to women ?
  \begin{items}
  \item America
  \item Greece
  \item* New Zealand
  \item Japan
  \end{items}

\question Nepal started elephant polo completion in the year:
  \begin{items}
  \item 1962 AD
  \item* 1982 AD
  \item 1992 AD
  \item 2002 AD
  \end{items}

\question Which among the following cities of Nepal is the first "Eco" city:
  \begin{items}
  \item Pokhara
  \item Madi
  \item Dharan
  \item* Bharatpur
  \end{items}

\question Which year did the SAARC university started to operate in
  \begin{items}
  \item 2008
  \item 2009
  \item* 2010
  \item 2011
  \end{items}

\question In how many countries of the world tigers are found?
  \begin{items}
  \item 5
  \item 9
  \item* 13
  \item 14
  \end{items}

\question When did EuroCup started ?
  \begin{items}
  \item* 1960
  \item 1964
  \item 1968
  \item 1990
  \end{items}

\question Which among the following cities does not comprise the Silk road?
  \begin{items}
  \item Bukhara, Uzbekistan
  \item Kabul, Afganistan
  \item* Jihadh, Saudi Arabia
  \item Aleppo, Syria
  \end{items}

\question ASEAN has how many member and supervisor nations?
  \begin{items}
  \item* 10, 2
  \item 9, 3
  \item 10, 4
  \item 10, 5
  \end{items}

\question The tallest peak of Chure range - Garba - has elevation of:
  \begin{items}
  \item* 1872 m
  \item 1730 m
  \item 1827 m
  \item 1890 m
  \end{items}

\question "It is our world" is the slogan of \fillin[][3cm] organization:
  \begin{items}
  \item WB
  \item WHO
  \item* UNO
  \item All of above
  \end{items}

\question BP Koirala inaugurated the BP highway in:
  \begin{items}
  \item* Shreekhandapur, Kavre
  \item Dumla, Sindhuli
  \item Bardibas, Mahottari
  \item Sindhuligadi, Sindhuli
  \end{items}

\question Nepal's first civil revolution is that of:
  \begin{items}
  \item* Biratnagar Jute mill
  \item Jhapa revolution
  \item Parcha revolution
  \item None
  \end{items}

\question BOAO forum for Asia summit organized in April 11, 2018 has elected as president:
  \begin{items}
  \item Xi Jin Ping
  \item Li Khichiang
  \item* Wan Ki Moon
  \item Hu Jin Tao
  \end{items}

\question Based on the office duration of individuals chairing house of representatives of Nepal which among following order is correct:
  \begin{enumerate}
  \item Ramchandra Poudel
  \item Damanath Dhungana
  \item Taranath Ranabhat
  \end{enumerate}
  \begin{items}
  \item 2, 3 and 1
  \item* 2, 1 and 3
  \item 1, 2 and 3
  \item 3, 2 and 1
  \end{items}

\question Banke, Bardiya, Kailali and Kanchanpur (a.k.a new nation) were returned to Nepal in:
  \begin{items}
  \item 1911 BS
  \item 1915 BS
  \item* 1917 BS
  \item 1919 BS
  \end{items}

\question According to village executive committee formation guidelines, in a village with 7 wards how many members form the village executive committe ?
  \begin{items}
  \item 13
  \item 14
  \item 15
  \item 16
  \end{items}

\question In the Human rights declaration of December 10, 1948, there are \fillin[][2cm] articles.
  \begin{items}
  \item 20
  \item* 30
  \item 32
  \item 40
  \end{items}

\question There are \fillin[][2cm] local bodies in Province 3.
  \begin{items}
  \item 116
  \item 117
  \item 118
  \item* 119
  \end{items}

\question Hulak Goshwara Adda was established during the regime of:
  \begin{items}
  \item Bhim Shamsher
  \item* Chandra Shamsher
  \item Juddha Shamsher
  \item Dev Shamsher
  \end{items}

\question "Rastriya Sabha Griha" was established with the assistance of \fillin[][3.5cm] while being designed by:
  \begin{items}
  \item* China, Gangadhar Bhatta
  \item India, Gangadhar Bhatta
  \item China, Atmakrishna Shrestha
  \item India, Atmakrishna Shrestha
  \end{items}

\question Which among the following statements are righly described below:
  \begin{enumerate}
  \item Chilime hydroelectricity project lies in Rasuwa district
  \item Chameliya hydroelectricity project lies in Darchula district
  \end{enumerate}
  \begin{items}
  \item* Both 1 and 2 are correct
  \item Both 1 and 2 are wrong
  \item 1 is correct and 2 is wrong
  \item 2 is correct and 1 is wrong
  \end{items}

\question 16th Aaha-Rara gold cup, 2019 finals was played between:
  \begin{items}
  \item* Nepal police club and Three star club
  \item Three star club and Manang marsyangdi club
  \item Thribhuwan army club and Nepal police
  \item Nepal police and Sahara club
  \end{items}

\question Which country does not have the same name for its country and capital?
  \begin{items}
  \item Luxemborg
  \item San marino
  \item Djibouti
  \item* Sicily
  \end{items}
  \begin{solution}
      All other are country and the names for its capital respectively with exception of Sicily which is the largest island in Mediterranean sea and one of the 20 regions of Italy.
  \end{solution}

\question Based on the area, the largest to smallest deserts of the world are:
  \begin{items}
  \item* Arabian, Gobi, Kalahari, Patagonia
  \item Gobi, Arabian, Kalahari, Patagonia
  \item Gobi, Arabian, Patagonia, Kalahari
  \item Arabian, Gobi, Patagonia, Kalahari
  \end{items}


\begin{solution}

Following are some of the deserts of the world. This does not enumerate the deserts by order of their sizes:

\begin{itemize}
    \item Arabian 899,618 sq miles. Spanning almost all of Arabian peninsula
    \item Atacama 600 mile long area rich in nitrate and copper deposits in northern chile
    \item Chihuahuan 139769 sq miles, Arizona and Mexico.
    \item Dasht-e Kavir 500 mile long by 200 mile wise in north-central Iran
    \item Dasht-e Lut 300 mile long by 200 mile wide in south-central Iran
    \item Death Valley 3300 sq miles, in California and Nevada
    \item Eastern (Arabian), Egypt. Between the Nile and Red sea extending south into Sudan
    \item Gibson, 60232 sq miles in the interior of western Australia
\end{itemize}

\end{solution}

\question Which of the following states does not have identity of "Baise rajya"?
  \begin{items}
  \item Jahari
  \item Dullu
  \item Bilashpur
  \item* Khanchi
  \end{items}

\question Which among the following matches is incorrect ?
  \begin{table}[h]
  \centering
  \begin{tabular}{lll}
    \textbf{Landmark} & & \textbf{Location} \\[2mm]
    1. Laddakh range & & Asia \\
    2. Andes range & & South america \\
    3. Alps range & & North america \\
    4. Karakoram range & & Asia \\
  \end{tabular}
  \end{table}
  \begin{items}
  \item 1 and 2 are incorrect
  \item 2 and 3 are incorrect
  \item Only 1 is incorrect
  \item* Only 3 is incorrect
  \end{items}

\question Which conference embodied the theme that "Women's rights are human rights"?
  \begin{items}
  \item The World Conference on Human Rights held by the United Nations in Vienna, Austria, on 14 to 25 June 1993.
  \item First World Women Conference, Mexico, 1975
  \item Second World Women Conference, Copenhegan, 1980
  \item Fourth World Women Conference, Beijing, 1995
  \end{items}

\question What is the correct order based on chronology of establishment ?
  \begin{items}
  \item Radio Nepal, Nepal Telecommunication Corporation, Gorkhapatra, Nepal Television
  \item* Gorkhapatra, Radio Nepal, Nepal Telecommunication Corporation, Nepal Television
  \item Nepal Telecommunication Corporation, Radio Nepal, Gorkhapatra, Nepal Television
  \item Gorkhapatra, Nepal Telecommunication Corporation, Radio Nepal, Nepal Television
  \end{items}

\question "For a living planet" is the motto of:
  \begin{items}
  \item IUCN
  \item UNEP
  \item* WWF
  \item UN
  \end{items}

\question "An agenda for peace" was presented by:
  \begin{items}
  \item* Butroes Butroes Gholi
  \item Kolfi Annan
  \item Wan Ki Moon
  \item None of above
  \end{items}

\question When is International day of Non-voilence observed ?
  \begin{items}
  \item* October 2
  \item October 3
  \item October 4
  \item October 5
  \end{items}

\question Cockpit of europe is:
  \begin{items}
  \item France
  \item Germany
  \item Serbia
  \item* Belgium
  \end{items}

\question Which of the following countries have not been named after river ?
  \begin{items}
  \item Paraguay
  \item Jambia
  \item* Sirya
  \item Niger
  \end{items}

\question Which district does not have two submetropolitan city ?
  \begin{items}
  \item Sunsari
  \item* Rupandehi
  \item Bara
  \item Dang
  \end{items}

\question Patagonia desert is in:
  \begin{items}
  \item China and congo
  \item Chile and Argentina
  \item* China and Kazakistan
  \item Argentina and Brasil
  \end{items}

\question Which of the following recognitions is also known as Nobel prize on environment
  \begin{items}
  \item Global 500
  \item* Goldman environment prize
  \item Galante conservation award
  \item Green peace prize
  \end{items}

\question Which among the following articles of Nepalese consitution was the first amendment made ?
  \begin{items}
  \item Article 43
  \item Article 84
  \item Article 167
  \item Article 283
  \end{items}

\question Katuwal lake is in:
  \begin{items}
  \item Okhaldhunga
  \item Chitwan
  \item* Kathmandu
  \item Lamjung
  \end{items}

\question Nagarkot, a tourism site situated in Bhaktapur, is \fillin[][2cm] masl.
  \begin{items}
  \item 2250
  \item* 2175
  \item 2275
  \item 2165
  \end{items}

\question Gopal dynasty had the capital in:
  \begin{items}
  \item* Gokarna
  \item Matatirtha
  \item Godabari
  \item Bouddha
  \end{items}

\question Folkland islands were the cause of battle in:
  \begin{items}
  \item* 1981
  \item 1983
  \item 1982
  \item 1984
  \end{items}

\question Which article of the constitution of Nepal contains provision of constitutional amendment ?
  \begin{items}
  \item* Article 274
  \item Article 243
  \item Article 285
  \item Article 266
  \end{items}

\question "Point of program" is associated with:
  \begin{items}
  \item Road construction
  \item Agricultural production management
  \item Organizational restructuring
  \item* Foreign aid
  \end{items}

\question "Village return" campaign was implemented in:
  \begin{items}
  \item 2020
  \item 2021
  \item 2022
  \item* 2024
  \end{items}

\question Jhimrukh hyrdoelectricity project is situated in:
  \begin{items}
  \item* Pyuthan
  \item Salyan
  \item Dailekh
  \item Syangja
  \end{items}

\question What is the size of provincial assembly in Province 7 ?
  \begin{items}
  \item* 53 (32 directly elected, 21 proportional)
  \item 54 (32 directly elected, 22 proportional)
  \item 53 (34 directly elected, 19 proportional)
  \item 53 (30 directly elected, 34 proportional)
  \end{items}

\question The provision regarding president and vice president that each of both should be of different geneder and of a different community is installed in constitution of Nepal's article:
  \begin{items}
  \item 70
  \item 71
  \item 72
  \item* 73
  \end{items}

\question Which of the following is wrong ?
  \begin{items}
  \item The district with 52 ponds and 53 lakes $\longrightarrow$ Kaski
  \item The gateway of Sagarmatha $\longrightarrow$ Namchebazzar
  \item The gateway of Nepal $\longrightarrow$ Birgunj
  \item Switzerland of Nepal $\longrightarrow$ Jiri
  \end{items}

\question When did the first paperless meeting of minestrial cabinet took place ?
  \begin{items}
  \item 2075, Shrawan 5
  \item* 2075, Shrawan 10
  \item 2075, Shrawan 12
  \item 2075, Shrawan 15
  \end{items}

\question When did Nepal signed the convention on the Elimination of all forms of Discrimination Against Women ?
  \begin{items}
  \item* 1991
  \item 1992
  \item 1993
  \item 1994
  \end{items}

  \begin{solution}
  The Convention on the Elimination of all Forms of Discrimination Against Women (CEDAW) is an international treaty adopted in 1979 by the United Nations General Assembly. Described as an international bill of rights for women, it was instituted on 3 September 1981 and has been ratified by 189 states.
  \end{solution}

\question Where is the UN university located ?
  \begin{items}
  \item America
  \item Japan
  \item* India
  \item France
  \end{items}

\question Which district is regarded as the centre of Deuki custom ?
  \begin{items}
  \item Achham
  \item* Baitadi
  \item Bajura
  \item Dadeldhura
  \end{items}

\question The length of Karnali bridge is:
  \begin{items}
  \item* 500m
  \item 606m
  \item 507m
  \item 608m
  \end{items}

\question What does honey mostly contain ?
  \begin{items}
  \item Protein
  \item Vitamin B
  \item* Carbohydrate
  \item Sugar
  \end{items}

\question When did Stephen Hawking die ?
  \begin{items}
  \item* March 14, 2018
  \item March 15, 2018
  \item March 16, 2018
  \item March 17, 2018
  \end{items}

\question Which of the facts below is true ?
  \begin{enumerate}
  \item Nepal and France tied diplomacy in 1949, April 20.
  \item France is a permanent member of UNO
  \item Nepal and France agreed in increasing support of France in civil aviation of Nepal in Falgun, 2054.
  \item Junga Bahadur Rana visited France.
  \end{enumerate}
  \begin{items}
  \item* All of above
  \item 2, 3 and 4
  \item 1, 3 and 4
  \item 2, 1 and 4
  \end{items}

\question For the regional alliances given below, find the correct order based on the longitude lowest to highest (East to West)
  \begin{items}
  \item EU, ASEAN (Association of Southeast Asian Nations), SAARC, OPEC (Organization of the Petroleum Exporting Countries)
  \item* ASEAN, SAARC, OPEC, EU
  \item SAARC, ASEAN, EU, OPEC
  \item OPEC, EU, ASEAN, SAARC
  \end{items}

\question Who is the first Chief secretary of Nepal ?
  \begin{items}
  \item Hari Prasad Pradhan
  \item Anerudra Prasad Singh
  \item* Chandra Bahadur Thapa
  \item Bhes Bahadur Thapa
  \end{items}

\question Which currency is also known as paper gold ?
  \begin{items}
  \item American dollar
  \item Sterling pound
  \item* SDR
  \item Euro
  \end{items}

\question Which countries in order approved the Nepal's proposition of being declared peace zone ?
  \begin{items}
  \item* China, Pakistan, North Korea, South Korea, Bangladesh
  \item China, North Korea, South Korea, Pakistan, Bangladesh
  \item China, Pakistan, South Korea, North Korea, Bangladesh
  \item China, Pakistan, Bangladesh, North Korea, South Korea
  \end{items}

\question Who is the fist elected chief minster of Nepal (provincial)
  \begin{items}
  \item Sherdhan Rai
  \item Lalbabu Raut
  \item Sankar Pokhrel
  \item* Dormani Poudel
  \end{items}

\question What is the height of TIA above sea level ?
  \begin{items}
  \item 1355
  \item 1350
  \item* 1337
  \item 1400
  \end{items}

\question When did the Betrawati treaty took place among Nepal, China and Tibet ?
  \begin{items}
  \item 1843 BS
  \item 1845 BS
  \item 1846 BS
  \item* 1849 BS
  \end{items}

\question Which of the following sayings is incorrect ?
  \begin{items}
  \item Nepal's major food crop is Rice
  \item Biratnagar Jute mill is the first industry ever established in Nepal
  \item* Nepal is WTO's 157th member
  \item World population day is celebrated in July 11, every year
  \end{items}

\question Which of the following caste does not come under marginalized group ?
  \begin{items}
  \item Sunuwar
  \item Tamang
  \item Dura
  \item* Chepang
  \end{items}

\question Considering the points below, select which applies.
  \begin{enumerate}
  \item Nepal's easternmost point lies in Taplejung district
  \item Nepal's westernmost point lies in Dodhara of Kanchanpur district
  \item Nepal's northernmost point lies in Changla of Humla district
  \item Nepal's southernmost point lies in Lodhabari of Jhapa district
  \end{enumerate}

  \begin{items}
  \item 1, 2 and 3 are correct
  \item 1, 3 and 4 are correct
  \item 1, 4 and 2 are correct
  \item* All are correct
  \end{items}

\question Which of the following matches is incorrect ?
  \begin{items}
  \item* Nepal's first king female embassador is Bindeshwori Shah
  \item Nepal was represented in Sugauli Treaty by Gajaraj Mishra
  \item It was decided to state Government of Nepal instead of His Majesty's Government in 15 Baisakh, 2063
  \item Constitutional assembly election 2 was conducted in Mangsir 4, 2070
  \end{items}

\question Bull fighting is the National game of:
  \begin{items}
  \item Italy
  \item Poland
  \item* Spain
  \item Finland
  \end{items}

\question Which country lost in Balkan War ?
  \begin{items}
  \item* Turkey
  \item Serbia
  \item Greece
  \item Albania
  \end{items}

\question Which of the following sayings is incorrect ?
  \begin{items}
  \item The book that enlists endangered species worldwide is called Red Book.
  \item The Red Book came into practice since 1966 AD
  \item* Currently Nepal is implementing Climate Change Policy, 2072
  \item Greenhouse effect was discovered in 1926 AD
  \end{items}

\question Which of the following match is incorrect ?
  \begin{items}
  \item World Environment Day $\longrightarrow$ June 5
  \item World Ocean Day $\longrightarrow$ June 8
  \item World Day against Child Labor $\longrightarrow$ June 14
  \item* World Refugee Day $\longrightarrow$ June 22
  \end{items}

\question Which among the following is not related to Industrial sector ?
  \begin{items}
  \item Dairy Development Corporation
  \item Hetauda Cement Industries
  \item Nepal Medicine Limited
  \item* Agriculture Input Company Limited
  \end{items}

\question SAARC summit and venues are given below. Identify the mismatch.
  \begin{items}
  \item* 14th $\longrightarrow$ Dhaka
  \item 15 $\longrightarrow$ Colombo
  \item 16 $\longrightarrow$ Thimpu
  \item 17 $\longrightarrow$ Addau Sahar
  \end{items}

\question Who among the following personalities were jailed while being declared recipient of the Nobel prize ?
  \begin{items}
  \item Carl von Ossietzky
  \item Aung San Suu Kyi
  \item Liu Xiaobo
  \item* All of above
  \end{items}

\question Since when did public transport system started in Nepal ?
  \begin{items}
  \item 1994 BS
  \item 1996 BS
  \item 1998 BS
  \item 1999 BS
  \end{items}

\question Consider the following and select all that apply.
  \begin{enumerate}
  \item Madam Curie received the 1903 Nobel prize in Physics
  \item Madam Curie shared the prize with Henri Becquerel for discovery fo Radioactivity.
  \end{enumerate}

  \begin{items}
  \item 1 is correct
  \item 2 is correct
  \item* Both are correct
  \item None are correct
  \end{items}

\question Which among following ASEAN nations is landlocked ?
  \begin{items}
  \item Malasiya
  \item Philippines
  \item* Laos
  \item Brunei
  \end{items}

\question Study and distinguish among following.
  \begin{enumerate}
  \item Amazon river originates from Andes range
  \item Volga is the biggest river of Europe
  \item Nile, the longest river of Africa, flows only in Egypt
  \item Mississipi-Missouri is the largest river of North America
  \end{enumerate}
  \begin{items}
  \item 1 and 2 are correct
  \item 1, 2 and 3 are correct
  \item All are correct
  \item* Only 3 is correct
  \end{items}

\question Which among the following planets takes the longest in revolving around the sun ?
  \begin{items}
  \item Jupiter
  \item Saturn
  \item Uranus
  \item* Neptune
  \end{items}

\question Based on geography Asia is divided into:
  \begin{items}
  \item 9 parts
  \item 7 parts
  \item 6 parts
  \item* 5 parts
  \end{items}

\question Amargadhi is in:
  \begin{items}
  \item Darchula district
  \item Bajhang district
  \item Bajura district
  \item* Dadeldhura district
  \end{items}

\question Louis-XVI of france was sentenced for death in:
  \begin{items}
  \item 1789 AD
  \item 1791 AD
  \item* 1793 AD
  \item 1795 AD
  \end{items}

\question \fillin[][2cm] used to only feed himself after all ate.
  \begin{items}
  \item Pratap Malla
  \item* Mahendra Malla
  \item Sivasingha Malla
  \item Ananta Malla
  \end{items}

\question "Chyabung" dance is a popular dance of:
  \begin{items}
  \item* Limbu
  \item Sherpa
  \item Lepcha
  \item Dhimal
  \end{items}

\question Who was the prime minister of Nepal during second world war ?
  \begin{items}
  \item Bir Samsher
  \item Juddha Samsher
  \item* Chandra Samsher
  \item Dev Samsher
  \end{items}

\question Ministry of Tourism was established in:
  \begin{items}
  \item 2032
  \item* 2033
  \item 2034
  \item 2035
  \end{items}

\question Bado tribes of Nepal are known by name:
  \begin{items}
  \item Tharu
  \item Chepang
  \item* Meche
  \item Rautey
  \end{items}

\question Which instrument is used for measuring air pressure ?
  \begin{items}
  \item* Manometer
  \item Ampere
  \item Bushels
  \item Lactometer
  \end{items}

\question A-1 satellite was launched by:
  \begin{items}
  \item* France
  \item Germany
  \item Japan
  \item Britain
  \end{items}

\question Which language got recognized the latest in the UN
  \begin{items}
  \item English
  \item* Arabian
  \item French
  \item Russian
  \end{items}

\question The diameter of the Earth is:
  \begin{items}
  \item 40800 km
  \item 40600 km
  \item 40400 km
  \item 40200 km
  \end{items}

\question Which is the biggest volcanic mountain of below:
  \begin{items}
  \item Himalayan range
  \item Mount K2
  \item Mainaloba
  \item Ural
  \end{items}

\question Based on area, which series represents the smallest to largest countries ?
  \begin{items}
  \item Russia, Canada, India, Australia, Brasil, China
  \item Russia, Canada, China, India, Australia, Brasil
  \item Russia, Canada, China, Brasil, Australia, India
  \item Russia, Canada, China, Australia, India, Brasil
  \end{items}

\question Nepal's border is adjoining with \fillin[][2cm] Indian states.
  \begin{items}
  \item 4
  \item 5
  \item 6
  \item 7
  \end{items}

\question Tilicho lake is \fillin[][2cm] masl.
  \begin{items}
  \item 3750
  \item 4319
  \item 4650
  \item 4919
  \end{items}

\question There are \fillin[][2cm] species of turtoise in Nepal.
  \begin{items}
  \item 12
  \item 14
  \item 10
  \item 16
  \end{items}

\question There were \fillin[][2cm] states in USA when it became soverign.
  \begin{items}
  \item 9
  \item 11
  \item 13
  \item 15
  \end{items}

\question Pashupatinath temple was conceived by:
  \begin{items}
  \item Prachandadev
  \item Yognarendra Malla
  \item Shankardev
  \item Nyayadev
  \end{items}

\question Which tribes believes that it is sinful to touch money ?
  \begin{items}
  \item Kusunda
  \item Thami
  \item Raute
  \item Chepang
  \end{items}

\question BBC started broadcast in Nepali since:
  \begin{items}
  \item 1967 June 7
  \item 1967 March 7
  \item 1967 July 7
  \item 1967 April 7
  \end{items}

\question Who was the king of Gorkha before Drabya Shah conquered ?
  \begin{items}
  \item Karki
  \item Bogati
  \item Khadka magar
  \item Thapa magar
  \end{items}

\question Which country has the highest literacy rate among the SAARC countries ?
  \begin{items}
  \item India
  \item Afganistan
  \item Maldives
  \item Bangladesh
  \end{items}

\question What is the word count of Nepalese national anthem.
  \begin{items}
  \item 26
  \item 36
  \item 46
  \item 56
  \end{items}

\question Mahendra cave is in:
  \begin{items}
  \item Kaski
  \item Pyuthan
  \item Taplejung
  \item Khotang
  \end{items}

\question Which is the nearest mountain to Kathmandu ?
  \begin{items}
  \item Gaurishankar
  \item Ganesh
  \item Machhapuchre
  \item Jugal
  \end{items}

\question Consider following statements and select which is true.
  \begin{items}
  \item When France was undergoing national revolution Louis XVI was the king.
  \item When France was undergoing national revolution Marie Antoniete was the queen.
  \item Both king and queen were guillotened for death sentence in 1994
  \item Voltaire argued that corrupted churches should be destroyed.
  \end{items}

\question The last king of Bhaktapur was:
  \begin{items}
  \item Bhupatindra Malla
  \item Pratap Malla
  \item Jayasthiti Malla
  \item Ranajit Malla
  \end{items}

\question Who led the battle of Nalapani ?
  \begin{items}
  \item Bhimsen Thapa
  \item Kalu Pandey
  \item Balbhadra Kunwar
  \item Gagasingh
  \end{items}

\question Lamosanghu-Jiri road section was constructed in assistance of:
  \begin{items}
  \item Japan
  \item Switzerland
  \item Germany
  \item Britain
  \end{items}

\question Who discovered Saturn ?
  \begin{items}
  \item Albert Einstein
  \item Gallileo
  \item Yuri Gagrin
  \item Archimedes
  \end{items}

\question Mention if the given statement is TRUE or FALSE.
  \begin{parts}
  \part There are 5 soil types found in Nepal. \hfill (T/F)
  \part Red silty type soil is suitable for Fingermillet cultivation. \hfill (T/F)
  \part Nepal initiated organized approach for soil conservation since 2002. \hfill (T/F)
  \part Afforestation was practiced in Nepal for the first time in 2002 BS. \hfill (T/F)
  \end{parts}

\question UNO has declared to celebrate year 2017 as:
  \begin{items}
  \item Poverty alleviation
  \item Year of sustainable tourism for development
  \item Sustainable development and peace
  \item Youth for development
  \end{items}

\question In which chapter ( \textit{Dhara}) of Nepalese constitution, there is statement that \textit{Madhesi}, \textit{Tharu} and \textit{Muslim} comission will be reconsidered after 10 years by assembly house:
  \begin{items}
  \item Dhara 264
  \item Dhara 266
  \item Dhara 265
  \item Dhara 267
  \end{items}

\question Which country was the first to switch off Radio Network ?
  \begin{items}
  \item Norway
  \item US
  \item Australia
  \item Denmark
  \end{items}

\question According to World Economic Forum, 2016, which country is regarded to have most inclusive development ?
  \begin{items}
  \item Lithuania
  \item Sweden
  \item Mauritiana
  \item Denmark
  \end{items}

\question Like NEPSE for Nepal, India has:
  \begin{items}
  \item ISE
  \item INE
  \item BSE
  \item INX
  \end{items}

\question Which among the following are UN peace keeping missions (multiple or none)
  \begin{items}
  \item UNMIL
  \item UNFIL
  \item UNMIN
  \item MINUSTAH
  \end{items}

\question Daniel Ortega has recently become president for third time of:
  \begin{items}
  \item Senegal
  \item Nicaragua
  \item Lithuania
  \item Mauritiana
  \end{items}

\question "The people's president: Dr. APJ Abdul Kalam" was written by:
  \begin{items}
  \item Hamid Ansari
  \item SM Khan
  \item Meghna Pant
  \item Akshaya Mukul
  \end{items}

\question Group of 77 is currently chaired by which country?
  \begin{items}
  \item Thailand
  \item Figi
  \item Equador
  \item Algeria
  \end{items}

\question International tuberculosis day is celebrated in remembrance of:
  \begin{items}
  \item Abraham Lincoln
  \item Mahatma Gandhi
  \item Nelsen Mandela
  \item Edward Joner
  \end{items}

\question When was the first Nepal Investment Summit organized ? (More in GK elaborate part)

\begin{items}
\item 1999
\item 1991
\item* 1992
\item 1994
\end{items}

\question When was the first Nepal Infrastructure Summit organized, in collaboration of Confederation of Nepalese Industries and Nepal Government  ? (More on GK elaborate part)

\begin{items}
\item Bhadra 15-16, 2071
\item Ashoj 25-26, 2071
\item* Kartik 25-26, 2071
\item Mangsir 1-2, 2071
\end{items}

\end{questions}





\subsection*{\fullwidth{\Large \centering \textbf{General Knowledge Multiple Choice: Accounting}}}
\begin{questions}

\question Generally what should be the ratio of internal rate of return and cost of capital rate ?
  \begin{items}
  \item Less than cost of capital rate
  \item More than cost of capital rate
  \item Equal of cost of capital rate
  \item None of the above
  \end{items}

\question Identify the importance of capital budgeting form the list given below:
  \begin{items}
  \item Long lasting impact
  \item Involves substantial amount of funds
  \item Decision is not flexible
  \item Ignore current assets
  \end{items}

\question What is called a relation between earning before tax and earning before interest and tax ?
  \begin{items}
  \item Financial leverages
  \item Combined leverages
  \item Operating leverages
  \item None of the above
  \end{items}

\question Financial statements are also known as ... statements.
  \begin{items}
  \item Behavioral
  \item Historical
  \item Research
  \item Preliminary
  \end{items}

\question Which of the following is correct relating to basic rules of Double entry accounting ?
  \begin{items}
  \item Should write debit on the right and credit on the left
  \item For every amount of debit there must be same amount of credit
  \item Debit and credit amount need not be equal
  \item None of the above
  \end{items}

\question Which act constitutes Accounting Standard Board:
  \begin{items}
  \item Fiscal act
  \item Audit act
  \item Appropriation act
  \item Nepal chartered accountant institutue act
  \end{items}

\question What status shows in balance sheet of company:
  \begin{items}
  \item Status of market management
  \item Status of profit and loss
  \item Status of capital and assets
  \item Status of cost of production
  \end{items}

\question Which amount does not consist of financial statements of company:
  \begin{items}
  \item Balance sheet
  \item Profit and loss account
  \item Cost account
  \item None of the above
  \end{items}

\question Goodwill is:
  \begin{items}
  \item Fictious asset
  \item Tangible asset
  \item Intangible asset
  \item Current asset
  \end{items}

\question Who can liquidate a company:
  \begin{items}
  \item Law
  \item Person
  \item Government
  \item None of the above
  \end{items}

\question Identify true or false in following statements
  \begin{enumerate}
  \item The main objective of the current company act is to make the incorporation operation and administration of companies much easeier, simpler and more transparent.
  \item According to company act, 2063 company means an organization incorporated under this act.
  \end{enumerate}
  \begin{items}
  \item 1 is true but 2 is false
  \item Both statements are true
  \item Both statements are false
  \item 1 is false but 2 is true
  \end{items}

\question Identify True or False in following statements:
  \begin{enumerate}
  \item Every company should maintain its accounts according to the double entry accounting system.
  \item Every company should maintain its accounts according to the single entry accounting system.
  \end{enumerate}

  \begin{items}
  \item 1 is true but 2 is false
  \item 1 is false but 2 is true
  \item Both statements are true
  \item Both statements are false
  \end{items}

\question Reason for business failure does not include the following aspects:
  \begin{items}
  \item Lack of sufficient fund
  \item Frequent change in governemnt policies
  \item Lack of quality products
  \item Managerial efficiency
  \end{items}

\question Identify True or False in following statements:
  \begin{enumerate}
  \item The general meeting of a company shall be annual and biannual.
  \item The general meeting of a company shall be annual and the extra ordinary.
  \end{enumerate}
  \begin{items}
  \item 1 is true but 2 is false
  \item 1 is false but 2 is true
  \item Both statements are true
  \item Both statements are false
  \end{items}

\question According to Nepal public sector accounting system, which of the following defines cash basis ?
  \begin{items}
  \item Cash flows
  \item Cash receipts
  \item Cash payments
  \item A basis of accounting that recognizes transactions and other events only when cash is received or paid
  \end{items}

\question Which of the following defines subsidiary company (According to the company act)?
  \begin{items}
  \item Private company
  \item Public company
  \item A company owned by a holding company
  \item Listed company
  \end{items}

\question Identify True or False in following statements:
  \begin{enumerate}
  \item The audit committee of a company should consist of at least three members.
  \item Chapter 10 of the current company act described about voluntary liquidation of a company
  \end{enumerate}
  \begin{items}
  \item 1 is true but 2 is false
  \item 1 is false but 2 is true
  \item Both statements are true
  \item Both statements are false
  \end{items}

\question The accounting system provides ... and ... financial information to interested parties.
  \begin{items}
  \item Relevant and reliable
  \item Efficient and effective
  \item Inclusive and equitable
  \item Praticipative and responsible
  \end{items}

\question Identify true or false in following statements:
  \begin{enumerate}
  \item Company should maintain its accounts in the English or the Nepali language.
  \item Company should maintain its accounts only in the Nepali language.
  \end{enumerate}
  \begin{items}
  \item 1 is true but 2 is false
  \item 1 is false but 2 is true
  \item Both statements are true
  \item Both statements are false
  \end{items}

\question Match the contents of the following and select the correct answer:
  \begin{table}[h]
  \centering
  \begin{tabular}{lll}
    a. Financial accounting & & 1. Receipt and payment account \\[2mm]
    b. Cost accounting & & 2. ABC analysis \\
    c. Management accounting & & 3. BEP analysis \\
    d. Accounting for professional & & 4. Profit and loss account \\
  \end{tabular}
  \end{table}
  \begin{items}
  \item a-4, b-2, c-3, d-1
  \item a-1, b-4, c-2, d-3
  \item a-1, b-3, c-4, d-2
  \item a-2, b-3, c-1, d-4
  \end{items}

\question On the basis of following information determine intrinsic value per share and select correct answer:
  \begin{table}[h]
  \centering
  \begin{tabular}{lll}
    Description & Vendor company & Purchasing company \\[2mm]
    Total assets (NRs) & 500000 & 12000000 \\
    Trade liabilities (NRs) & 170000 & 300000 \\
    Share value each costing Rs 100 & 150000 & 300000 \\
  \end{tabular}
  \end{table}
  (Information: The holder of every three shares in vendor company were to receive two shares in purchasing company plus cash as necessary to adjustment.)
  \begin{items}
  \item 220 for vendor company and 300 for purchasing company
  \item 250 for vendor company and 320 for purchasing company
  \item 300 for vendor company and 300 for purchasing company
  \item 300 for vendor company and 350 for purchasing company
  \end{items}

\question In which date prevailing company act comes into force ?
  \begin{items}
  \item 2052
  \item 2053
  \item 2054
  \item 2055
  \end{items}

\question Match the contents of the following and select the correct answer:
  \begin{table}[h]
  \centering
  \begin{tabular}{lll}
    a. Liquidation & & 1. Company operation continues \\[2mm]
    b. Amalagamation & & 2. Jointly new company establishment \\
    c. Absorption & & 3. Internal reconstruction \\
    d. Reconstruction & & 4. Company registrar office \\
  \end{tabular}
  \end{table}
  \begin{items}
  \item a-4, b-2, c-1, d-3
  \item a-1, b-4, c-2, d-3
  \item a-1, b-3, c-4, d-2
  \item a-2, b-3, c-1, d-4
  \end{items}

\question In which of the following conditions company goes into liquidation ?
  \begin{items}
  \item By the order of Company registrar
  \item By the order of court
  \item By the decision of cabinet
  \item None of the above
  \end{items}

\question In which priority the payment of liquidation expenses lies ?
  \begin{items}
  \item First
  \item Second
  \item Third
  \item Fourth
  \end{items}

\question Which of the following account is not a requirement for professional men ?
  \begin{items}
  \item Receipt and payment A/C
  \item Income and expenditure A/C
  \item Balance sheet
  \item Profit and loss A/C
  \end{items}

\question Match the correct pair.
  \begin{table}[h]
  \centering
  \begin{tabular}{lll}
    a. Assets & & 1. Creditors \\[2mm]
    b. Liabilities & & 2. Ratio \\
    c. Equity & & 3. Interest income \\
    d. Revenue & & 4. Divident \\
  \end{tabular}
  \end{table}
  \begin{items}
  \item a-2, b-1, c-4, d-3
  \item a-1, b-4, c-2, d-3
  \item a-1, b-3, c-4, d-2
  \item a-2, b-3, c-1, d-4
  \end{items}

\question Accounting for profession includes all of the following account expect:
  \begin{items}
  \item Receipt and payment account
  \item Income and expenditure account
  \item Balance sheet
  \item Government account
  \end{items}

\question Which one is not the qualitative characteristic of financial statements ?
  \begin{items}
  \item Relevance
  \item Reliability
  \item Comparability
  \item Inclusiveness
  \end{items}

\question Generally accepted accounting principle GAAP includes all the following except:
  \begin{items}
  \item Business entity concept
  \item Going concern concept
  \item Cost concept
  \item Inefficiency concept
  \end{items}

\question The number of shareholders of a private company should not exceed ...
  \begin{items}
  \item Fifty
  \item Sixty
  \item Seventy
  \item Eighty
  \end{items}

\question Except as otherwise, the paid up capital of a public company should be a minimum of ...
  \begin{items}
  \item 10 million
  \item 20 million
  \item 30 million
  \item 40 million
  \end{items}

\question According to current company act, the term officer includes all the following except:
  \begin{items}
  \item Director
  \item Chief executive
  \item Manager
  \item Shareholder
  \end{items}

\question Every public company's Board of Directors should prepare its annual financial statements at least ... days prior to the holding of its annual general meeting.
  \begin{items}
  \item 15
  \item 20
  \item 25
  \item 30
  \end{items}

\question Who among following is the internal user of Accounting information ?
  \begin{items}
  \item Creditor
  \item Government
  \item Personnel
  \item Bank
  \end{items}

\question Which among the following is correct ? Generally stock value of the bank increases, when:
  \begin{enumerate}
  \item Profit of the bank increases
  \item In the condition of divident increment
  \item Risk reduction by reducing debt
  \item Increase of capital fund
  \end{enumerate}
  \begin{items}
  \item All are correct
  \item All are incorrect
  \item a and b are correct
  \item c and d are correct
  \end{items}

\question Ratio between bank's revenue and assets is called:
  \begin{items}
  \item Net creditor limit
  \item Net profit margin
  \item Net operation margin
  \item Bank performance
  \end{items}

\question Why trial balance is prepared in a company ?
  \begin{items}
  \item To identify profit/loss
  \item To examine mathematical accuracy of ledger
  \item To trace the cash balance
  \item None of the above
  \end{items}

\question What is amalgamation of companies ?
  \begin{items}
  \item Absorption of one company
  \item Reorganization of two or more companies
  \item Reconstruction of a company
  \item Reformation of a company
  \end{items}

\question 50\% or more of total capital of a company invested as share investment by another company is called ...
  \begin{items}
  \item Subsidiary company
  \item Holding company
  \item New company
  \item None of the above
  \end{items}

\question AMAN company purchased 8000 shares of BIMAN company at the rate of Rs 150 per share. Actual share value of BIMAN company is only Rs 100 per share. Capital gain of Rs 150000 is allotted to AMAN company. On the base of these information calculate the value of Goodwill.
  \begin{items}
  \item Rs 200000
  \item Rs 250000
  \item Rs 300000
  \item Rs 400000
  \end{items}

\question In which of the following condition name of the company need not be changed ?
  \begin{items}
  \item Amalgamation
  \item Absorption
  \item Reconstruction
  \item Internal reconstruction
  \end{items}

\question Which of the following transaction is not included in Consolidated Balance sheet ?
  \begin{items}
  \item Profit before take-over
  \item Profit after take-over
  \item Loss after take-over
  \item Inter company transaction
  \end{items}

\question Nath Co. Ltd sells its business to Anath Co. Ltd. The balance sheet of Nath Co. Ltd on the data is as follows:
  \begin{table}[h]
  \centering
  \begin{tabular}{llll}
    \textbf{Liabilities} & & \textbf{Assets} & \\[2mm]
    Issued capital & 150000 & Plant and machinery & 180000 \\
    Creditor & 100000 & Furniture & 30000 \\
    Workmen fund & 30000 & Cash & 30000 \\
    Debenture & 50000 & Debtors & 100000 \\
    Profit/loss & 15000 & Preliminary expenses & 5000 \\
    Total & 345000 & Total & 345000
  \end{tabular}
  \end{table}
  Which is correct purchase consideration under net asset method:
  \begin{items}
  \item 220000
  \item 230000
  \item 240000
  \item 250000
  \end{items}

\question Identify True or False in following statements:
  \begin{enumerate}
  \item Net income = Total revenue - Total expenditure
  \item Net income = Total revenue - Total debt
  \end{enumerate}
  \begin{items}
  \item 1 is true but 2 is false
  \item Both statements are true
  \item Both statements are false
  \item 1 is false but 2 is true
  \end{items}

\question Match the following:
  \begin{table}[h]
  \centering
  \begin{tabular}{lll}
    Setting account standard & Auditing standard board \\[2mm]
    Setting auditing standard & Accounting standard board \\
    Developing of accounting profession & Auditor general \\
    Development of accounting profession & Institute of charter accounting of nepal \\
  \end{tabular}
  \end{table}
  \begin{items}
  \item a-3, b-4, c-2, d-1
  \item a-2, b-1, c-4, d-3
  \item a-4, b-3, c-2, d-1
  \item a-1, b-2, c-3, d-4
  \end{items}

\question Identify the most obvious users of accounting information person is also called ...
  \begin{items}
  \item Government and its agencies
  \item Investors and lenders
  \item Development partners
  \item Political parties
  \end{items}

\question Accounting equation prepared by professional person is called ...
  \begin{items}
  \item Capital equation
  \item Balance sheet equation
  \item Profit and loss
  \item Liability equation
  \end{items}

\question As per income tax act which one of following is not professional person ?
  \begin{items}
  \item Doctor
  \item Auditor
  \item Lawyer
  \item Farmer
  \end{items}

\question Read the following statements and identify the correct and incorrect alternative:
  \begin{enumerate}
  \item Capital income and expenditures are not included in income and expenditure account
  \item Balance sheet is position statement of business
  \item Income and expenditure account based on real account
  \end{enumerate}
  \begin{items}
  \item 1 and 2 are correct but 3 is incorrect
  \item 1 is correct but 2 and 3 are incorrect
  \item 2 is correct but 1 and 2 are incorrect
  \item 1 and 2 are correct but 3 is incorrect
  \end{items}

\question What types of transactions between holding and subsidiary company are not shown in consolidated balance sheet ?
  \begin{items}
  \item Profit after holding company
  \item Loss after holding company
  \item Profit before holding company
  \item Inter company transaction
  \end{items}

\question What is the name of profit before making investment in subsidiary company by holding company ?
  \begin{items}
  \item Capital profit
  \item Revenue profit
  \item Pre-acquisition profit
  \item Unrealized profit
  \end{items}

\question What is the additional amount of money paid by the holding company ?
  \begin{items}
  \item Premium
  \item Interest
  \item Discount
  \item Goodwill
  \end{items}

\end{questions}
